The \href{https://code.fbi.h-da.de/pse_mayer_ss19/trunk/tree/master/Allgemein/Beispiele/CY8Ckit-149-BLE/I2C_Event.cydsn}{\texttt{ event structure}} has peen merged into this project.

Each sensor will have to create their own events as the measurements come in.

In order to implement a new event, you need to make sure there is a corresponding event type and a corresponding event subtype. The event type specifies which measurements are in the event while the event subtype indicates which sensor the measurements came from. To do this, open {\ttfamily \mbox{\hyperlink{event_8h_source}{event.\+h}}} and check for / add definitions for {\ttfamily E\+V\+T\+Y\+P\+\_\+\+M\+E\+A\+S\+U\+R\+E\+M\+E\+NT} \& {\ttfamily E\+V\+S\+U\+B\+T\+Y\+P\+\_\+\+S\+E\+N\+S\+OR}. You also need a {\ttfamily typedef} for a pointer to your event\+: {\ttfamily typedef s\+Evt\+Measurement\+\_\+t$\ast$ ps\+Evt\+Measurement\+\_\+t;}.

Next, check for a {\ttfamily typedef} for your event. If one does not exist, create one like this\+:


\begin{DoxyCode}{0}
\DoxyCodeLine{ \{C++\}}
\DoxyCodeLine{typedef struct\{}
\DoxyCodeLine{    // Event header}
\DoxyCodeLine{    sEvtHead\_t sEvtHead;      // Event type (has a length of 4 bytes)}
\DoxyCodeLine{    // Event payload}
\DoxyCodeLine{    uint32\_t measurement;     // Measurement}
\DoxyCodeLine{\} sEvtMeasurement\_t;}
\end{DoxyCode}


Then you need to declare two functions to actually create an event\+: {\ttfamily void$\ast$ event\+Write\+Measurement(const uint8 subtype, const uint32\+\_\+t measurement);} \& {\ttfamily void$\ast$ event\+Write\+Sensor\+Measurement(const uint32\+\_\+t measurement);}.

Now go into {\ttfamily \mbox{\hyperlink{event_8c}{event.\+c}}} and implement these two functions like this\+: 
\begin{DoxyCode}{0}
\DoxyCodeLine{ \{C++\}}
\DoxyCodeLine{void* eventWriteMeasurement(const uint8 subtype, const uint32\_t measurement) \{}
\DoxyCodeLine{    // Make sure there is enough free storage space in the event buffer for this event}
\DoxyCodeLine{    if(getFreeStorageSize() < sizeof(sEvtMeasurement\_t))\{}
\DoxyCodeLine{        // To avoid overflowing and subsequently crashing, don't write the event}
\DoxyCodeLine{        PRINTF("Event buffer full.\(\backslash\)r\(\backslash\)n");}
\DoxyCodeLine{        return pDataPtr;}
\DoxyCodeLine{    \}}
\DoxyCodeLine{}
\DoxyCodeLine{    // Create the event}
\DoxyCodeLine{    psEvtMeasurement\_t eventMeasurement = (psEvtMeasurement\_t)pDataPtr;}
\DoxyCodeLine{}
\DoxyCodeLine{    // Write the event header}
\DoxyCodeLine{    eventMeasurement->sEvtHead.evtType = EVTYP\_MEASUREMENT;         // Event type}
\DoxyCodeLine{    eventMeasurement->sEvtHead.evtSubtype = subtype;                // Event subtype}
\DoxyCodeLine{    eventMeasurement->sEvtHead.evtLen = sizeof(sEvtMeasurement\_t);  // Event length}
\DoxyCodeLine{}
\DoxyCodeLine{    // Write the event data}
\DoxyCodeLine{    eventMeasurement->measurement = measurement;}
\DoxyCodeLine{}
\DoxyCodeLine{    // Increment event pointer}
\DoxyCodeLine{    pDataPtr += eventMeasurement->sEvtHead.evtLen;}
\DoxyCodeLine{}
\DoxyCodeLine{    // Write EOF event to mark the end of all written events.}
\DoxyCodeLine{    // Everything in the buffer beyond this is free to write to and does not contain any usable data.}
\DoxyCodeLine{    writeEOF();}
\DoxyCodeLine{}
\DoxyCodeLine{    // Return the next free address for the next event}
\DoxyCodeLine{    return pDataPtr;}
\DoxyCodeLine{\}}
\DoxyCodeLine{}
\DoxyCodeLine{void* eventWriteSensorMeasurement(const uint32\_t measurement) \{}
\DoxyCodeLine{    return eventWriteMeasurement(EVSUBTYP\_SENSOR, measurement);}
\DoxyCodeLine{\}}
\end{DoxyCode}


Make sure you also include a way to display your new event on the terminal, in {\ttfamily void terminal\+Input\+Char(uint8 ch)}\+: 
\begin{DoxyCode}{0}
\DoxyCodeLine{ \{C++\}}
\DoxyCodeLine{case EVTYP\_MEASUREMENT: \{}
\DoxyCodeLine{    psEvtMeasurement\_t eventMeasurement = (psEvtMeasurement\_t)pCurrDataBuf;}
\DoxyCodeLine{    PRINTF("Measurement: \%d\(\backslash\)r\(\backslash\)n", eventMeasurement->measurement);}
\DoxyCodeLine{    break;}
\DoxyCodeLine{\}}
\end{DoxyCode}


The last thing then is to actually create an event with your sensor, by adding {\ttfamily event\+Write\+Sensor\+Measurement(measurement);} to your sensors measuring method.

If you did it correctly, you can start your measurements and then read them by displaying all events on the terminal.\hypertarget{autotoc_md19_autotoc_md21}{}\section{Sensors}\label{autotoc_md19_autotoc_md21}
\hypertarget{autotoc_md19_autotoc_md22}{}\subsection{Availability of all possible configurations of sensors via I2C}\label{autotoc_md19_autotoc_md22}
\tabulinesep=1mm
\begin{longtabu}spread 0pt [c]{*{2}{|X[-1]}|}
\hline
\PBS\centering \cellcolor{\tableheadbgcolor}\textbf{ Symbol  }&\PBS\centering \cellcolor{\tableheadbgcolor}\textbf{ Meaning   }\\\cline{1-2}
\endfirsthead
\hline
\endfoot
\hline
\PBS\centering \cellcolor{\tableheadbgcolor}\textbf{ Symbol  }&\PBS\centering \cellcolor{\tableheadbgcolor}\textbf{ Meaning   }\\\cline{1-2}
\endhead
-\/  &Not connected   \\\cline{1-2}
✔  &Working   \\\cline{1-2}
❌  &Not working   \\\cline{1-2}
\end{longtabu}
\hypertarget{autotoc_md19_autotoc_md23}{}\subsubsection{4800 Ω pull-\/up}\label{autotoc_md19_autotoc_md23}
\tabulinesep=1mm
\begin{longtabu}spread 0pt [c]{*{3}{|X[-1]}|}
\hline
\PBS\centering \cellcolor{\tableheadbgcolor}\textbf{ S\+H\+T31  }&\PBS\centering \cellcolor{\tableheadbgcolor}\textbf{ S\+G\+P30  }&\PBS\centering \cellcolor{\tableheadbgcolor}\textbf{ S\+P\+S30   }\\\cline{1-3}
\endfirsthead
\hline
\endfoot
\hline
\PBS\centering \cellcolor{\tableheadbgcolor}\textbf{ S\+H\+T31  }&\PBS\centering \cellcolor{\tableheadbgcolor}\textbf{ S\+G\+P30  }&\PBS\centering \cellcolor{\tableheadbgcolor}\textbf{ S\+P\+S30   }\\\cline{1-3}
\endhead
-\/  &-\/  &-\/   \\\cline{1-3}
-\/  &-\/  &✔   \\\cline{1-3}
-\/  &✔  &-\/   \\\cline{1-3}
-\/  &✔  &✔   \\\cline{1-3}
✔  &-\/  &-\/   \\\cline{1-3}
✔  &-\/  &✔   \\\cline{1-3}
✔  &✔  &-\/   \\\cline{1-3}
✔  &✔  &✔   \\\cline{1-3}
\end{longtabu}
\hypertarget{autotoc_md19_autotoc_md24}{}\subsubsection{No pull-\/up}\label{autotoc_md19_autotoc_md24}
\tabulinesep=1mm
\begin{longtabu}spread 0pt [c]{*{3}{|X[-1]}|}
\hline
\PBS\centering \cellcolor{\tableheadbgcolor}\textbf{ S\+H\+T31  }&\PBS\centering \cellcolor{\tableheadbgcolor}\textbf{ S\+G\+P30  }&\PBS\centering \cellcolor{\tableheadbgcolor}\textbf{ S\+P\+S30   }\\\cline{1-3}
\endfirsthead
\hline
\endfoot
\hline
\PBS\centering \cellcolor{\tableheadbgcolor}\textbf{ S\+H\+T31  }&\PBS\centering \cellcolor{\tableheadbgcolor}\textbf{ S\+G\+P30  }&\PBS\centering \cellcolor{\tableheadbgcolor}\textbf{ S\+P\+S30   }\\\cline{1-3}
\endhead
-\/  &-\/  &-\/   \\\cline{1-3}
-\/  &-\/  &❌   \\\cline{1-3}
-\/  &✔  &-\/   \\\cline{1-3}
-\/  &✔  &✔   \\\cline{1-3}
✔  &-\/  &-\/   \\\cline{1-3}
✔  &-\/  &✔   \\\cline{1-3}
✔  &✔  &-\/   \\\cline{1-3}
✔  &✔  &✔   \\\cline{1-3}
\end{longtabu}
\hypertarget{autotoc_md19_autotoc_md25}{}\subsection{$<$a href=\char`\"{}https\+://www.\+sensirion.\+com/en/environmental-\/sensors/particulate-\/matter-\/sensors-\/pm25/\char`\"{}$>$\+S\+P\+S30$<$/a$>$}\label{autotoc_md19_autotoc_md25}
The S\+P\+S30 sensor by \href{https://www.sensirion.com/en/}{\texttt{ Sensirion}} measures \href{https://en.wikipedia.org/wiki/Particulates}{\texttt{ particulates}} in the air by laser scattering.

\href{https://github.com/Sensirion/embedded-sps}{\texttt{ The Driver used}} is made by \href{https://www.sensirion.com/en/}{\texttt{ Sensirion}}.\hypertarget{autotoc_md19_autotoc_md26}{}\subsubsection{Pin Assignment\+:}\label{autotoc_md19_autotoc_md26}


\tabulinesep=1mm
\begin{longtabu}spread 0pt [c]{*{3}{|X[-1]}|}
\hline
\PBS\centering \cellcolor{\tableheadbgcolor}\textbf{ Pin Nr  }&\PBS\centering \cellcolor{\tableheadbgcolor}\textbf{ Type  }&\PBS\centering \cellcolor{\tableheadbgcolor}\textbf{ Connect   }\\\cline{1-3}
\endfirsthead
\hline
\endfoot
\hline
\PBS\centering \cellcolor{\tableheadbgcolor}\textbf{ Pin Nr  }&\PBS\centering \cellcolor{\tableheadbgcolor}\textbf{ Type  }&\PBS\centering \cellcolor{\tableheadbgcolor}\textbf{ Connect   }\\\cline{1-3}
\endhead
1  &V\+DD  &V\+DD   \\\cline{1-3}
2  &S\+DA  &6.\+5   \\\cline{1-3}
3  &S\+CL  &6.\+4   \\\cline{1-3}
4  &S\+EL  &G\+ND   \\\cline{1-3}
5  &G\+ND  &G\+ND   \\\cline{1-3}
\end{longtabu}
\hypertarget{autotoc_md19_autotoc_md27}{}\subsubsection{Usage}\label{autotoc_md19_autotoc_md27}
Initialise a struct {\ttfamily struct \mbox{\hyperlink{structsps30__measurement}{sps30\+\_\+measurement}} measurement} before measuring.

Start the sensor.

Measure the result with {\ttfamily S\+P\+S30\+\_\+\+Read\+\_\+\+Data(\&measurement, N\+U\+L\+L)}.

When the S\+P\+S30 is the only sensor connected to the board, make sure there is a pull-\/up resistor connected to the I2C sda \& scl lines.\hypertarget{autotoc_md19_autotoc_md28}{}\subsection{V\+O\+C\+Sens}\label{autotoc_md19_autotoc_md28}
The \href{https://www.glyn.de/Produkte/Sensoren/VOCSens}{\texttt{ V\+O\+C\+Sens board}} combines two sensors on one board, namely the \href{https://www.sensirion.com/en/environmental-sensors/humidity-sensors/digital-humidity-sensors-for-various-applications/}{\texttt{ S\+H\+T31}} \& the \href{https://www.sensirion.com/de/umweltsensoren/gassensoren/multipixel-gassensoren/}{\texttt{ S\+G\+P30}}, both are made by \href{https://www.sensirion.com/en/}{\texttt{ Sensirion}}.\hypertarget{autotoc_md19_autotoc_md29}{}\subsection{$<$a href=\char`\"{}https\+://www.\+sensirion.\+com/en/environmental-\/sensors/humidity-\/sensors/digital-\/humidity-\/sensors-\/for-\/various-\/applications/\char`\"{}$>$\+S\+H\+T31 Sensor$<$/a$>$}\label{autotoc_md19_autotoc_md29}
\href{https://github.com/Sensirion/embedded-sht}{\texttt{ The Driver used}} is made by Sensirion.\hypertarget{autotoc_md19_autotoc_md30}{}\subsubsection{Usage}\label{autotoc_md19_autotoc_md30}
Initialise the S\+H\+T31 Sensor.

Initialise two floats, one for the temperature and one for the humidity.

Then either initialise two pointers to those variables and pass those as arguments to the measuring function, or pass their addresses directly.

You can either use a function which blocks during the measurement, or a non-\/blocking one. In this case you have to make sure to let enough time pass for the sensor to complete the measurement, which takes 15ms.

The U\+A\+RT uses a baud rate of 115200.

The ° from x°C might not render correctly in the terminal.

Your terminal output should look something like this\+: 
\begin{DoxyCode}{0}
\DoxyCodeLine{Measured temperature: 24.152°C}
\DoxyCodeLine{Measured humidity:    50.531\%}
\DoxyCodeLine{Measured temperature: 24.165°C}
\DoxyCodeLine{Measured humidity:    50.521\%}
\DoxyCodeLine{Measured temperature: 24.136°C}
\DoxyCodeLine{Measured humidity:    50.543\%}
\DoxyCodeLine{Measured temperature: 24.152°C}
\DoxyCodeLine{Measured humidity:    50.531\%}
\DoxyCodeLine{...}
\end{DoxyCode}
\hypertarget{autotoc_md19_autotoc_md31}{}\subsubsection{Combining with a different project}\label{autotoc_md19_autotoc_md31}

\begin{DoxyItemize}
\item Copy $\ast$.c \& $\ast$.h into your project, but not main.\+c.
\item Copy both elements from Top\+Design into your project.
\item Assign pins according to your needs (this project uses 3.\+0 for I2C scl, 3.\+1 for I2C sda; 7.\+0 for U\+A\+RT rx \& 7.\+1 for U\+A\+RT tx)
\item In your main.\+c, include \mbox{\hyperlink{_s_h_t31_8h_source}{S\+H\+T31.\+h}}.
\item You need to initialise the sensor once using {\ttfamily G\+Y\+\_\+\+S\+H\+T31\+\_\+\+Init()}
\end{DoxyItemize}

To measure, use {\ttfamily G\+Y\+\_\+\+S\+H\+T31\+\_\+\+Measure(double$\ast$ temperature\+\_\+, double$\ast$ humidity\+\_\+, bool console\+\_\+out)} like this\+: 
\begin{DoxyCode}{0}
\DoxyCodeLine{ \{C++\}}
\DoxyCodeLine{double temperature;}
\DoxyCodeLine{double humidity;}
\DoxyCodeLine{}
\DoxyCodeLine{GY\_SHT31\_Measure(\&temperature, \&humidity, true);}
\end{DoxyCode}


Note that function blocks during the duration of the measurement.

You can also use {\ttfamily G\+Y\+\_\+\+S\+H\+T31\+\_\+\+Ready\+\_\+\+Measurement();} and {\ttfamily G\+Y\+\_\+\+S\+H\+T31\+\_\+\+Read\+\_\+\+Measurement(double$\ast$ temperature\+\_\+, double$\ast$ humidity\+\_\+, bool console\+\_\+out);} as a non-\/blocking alternative\+: 
\begin{DoxyCode}{0}
\DoxyCodeLine{ \{C++\}}
\DoxyCodeLine{double temperature;}
\DoxyCodeLine{double humidity;}
\DoxyCodeLine{}
\DoxyCodeLine{GY\_SHT31\_Ready\_Measurement();}
\DoxyCodeLine{CyDelay(15);}
\DoxyCodeLine{GY\_SHT31\_Read\_Measurement(\&temperature, \&humidity, true);}
\end{DoxyCode}
\hypertarget{autotoc_md19_autotoc_md32}{}\subsection{$<$a href=\char`\"{}https\+://learn.\+adafruit.\+com/adafruit-\/sgp30-\/gas-\/tvoc-\/eco2-\/mox-\/sensor/\char`\"{}$>$\+Adafruit S\+G\+P30$<$/a$>$}\label{autotoc_md19_autotoc_md32}
\hypertarget{autotoc_md19_autotoc_md33}{}\subsubsection{e\+C\+O2 (equivalent calculated carbon-\/dioxide)}\label{autotoc_md19_autotoc_md33}
concentration within a range of 400 to 60,000 parts per million (ppm) \href{https://ec.europa.eu/eurostat/statistics-explained/index.php/Glossary:Carbon_dioxide_equivalent}{\texttt{ https\+://ec.\+europa.\+eu/eurostat/statistics-\/explained/index.\+php/\+Glossary\+:\+Carbon\+\_\+dioxide\+\_\+equivalent}}\hypertarget{autotoc_md19_autotoc_md34}{}\subsubsection{T\+V\+O\+C (\+Total Volatile Organic Compound)}\label{autotoc_md19_autotoc_md34}
concentration within a range of 0 to 60,000 parts per billion (ppb) \href{https://www.umweltbundesamt.de/themen/gesundheit/umwelteinfluesse-auf-den-menschen/chemische-stoffe/fluechtige-organische-verbindungen\#textpart-1}{\texttt{ https\+://www.\+umweltbundesamt.\+de/themen/gesundheit/umwelteinfluesse-\/auf-\/den-\/menschen/chemische-\/stoffe/fluechtige-\/organische-\/verbindungen\#textpart-\/1}} 